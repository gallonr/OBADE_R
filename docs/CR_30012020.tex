% Options for packages loaded elsewhere
\PassOptionsToPackage{unicode}{hyperref}
\PassOptionsToPackage{hyphens}{url}
%
\documentclass[
]{article}
\usepackage{lmodern}
\usepackage{amssymb,amsmath}
\usepackage{ifxetex,ifluatex}
\ifnum 0\ifxetex 1\fi\ifluatex 1\fi=0 % if pdftex
  \usepackage[T1]{fontenc}
  \usepackage[utf8]{inputenc}
  \usepackage{textcomp} % provide euro and other symbols
\else % if luatex or xetex
  \usepackage{unicode-math}
  \defaultfontfeatures{Scale=MatchLowercase}
  \defaultfontfeatures[\rmfamily]{Ligatures=TeX,Scale=1}
\fi
% Use upquote if available, for straight quotes in verbatim environments
\IfFileExists{upquote.sty}{\usepackage{upquote}}{}
\IfFileExists{microtype.sty}{% use microtype if available
  \usepackage[]{microtype}
  \UseMicrotypeSet[protrusion]{basicmath} % disable protrusion for tt fonts
}{}
\makeatletter
\@ifundefined{KOMAClassName}{% if non-KOMA class
  \IfFileExists{parskip.sty}{%
    \usepackage{parskip}
  }{% else
    \setlength{\parindent}{0pt}
    \setlength{\parskip}{6pt plus 2pt minus 1pt}}
}{% if KOMA class
  \KOMAoptions{parskip=half}}
\makeatother
\usepackage{xcolor}
\IfFileExists{xurl.sty}{\usepackage{xurl}}{} % add URL line breaks if available
\IfFileExists{bookmark.sty}{\usepackage{bookmark}}{\usepackage{hyperref}}
\hypersetup{
  pdftitle={Compte-rendu du 30/01/2020},
  hidelinks,
  pdfcreator={LaTeX via pandoc}}
\urlstyle{same} % disable monospaced font for URLs
\usepackage[margin=1in]{geometry}
\usepackage{graphicx,grffile}
\makeatletter
\def\maxwidth{\ifdim\Gin@nat@width>\linewidth\linewidth\else\Gin@nat@width\fi}
\def\maxheight{\ifdim\Gin@nat@height>\textheight\textheight\else\Gin@nat@height\fi}
\makeatother
% Scale images if necessary, so that they will not overflow the page
% margins by default, and it is still possible to overwrite the defaults
% using explicit options in \includegraphics[width, height, ...]{}
\setkeys{Gin}{width=\maxwidth,height=\maxheight,keepaspectratio}
% Set default figure placement to htbp
\makeatletter
\def\fps@figure{htbp}
\makeatother
\setlength{\emergencystretch}{3em} % prevent overfull lines
\providecommand{\tightlist}{%
  \setlength{\itemsep}{0pt}\setlength{\parskip}{0pt}}
\setcounter{secnumdepth}{-\maxdimen} % remove section numbering

\title{Compte-rendu du \textbf{30/01/2020}}
\author{}
\date{\vspace{-2.5em}}

\begin{document}
\maketitle

\begin{quote}
Présent(e)s :\\
- Aurélie Garcia (AG)\\
- Léa Brieau (LB)\\
- Régis Gallon (RG)
\end{quote}

\emph{LB} a été intégrée dans la mailing-list du projet OBADE

\hypertarget{description-des-variables-et-homoguxe9nuxe9isation-des-libelluxe9s}{%
\subsubsection{\texorpdfstring{\textbf{1. Description des variables et
homogénéisation des
libellés}}{1. Description des variables et homogénéisation des libellés}}\label{description-des-variables-et-homoguxe9nuxe9isation-des-libelluxe9s}}

\begin{itemize}
\tightlist
\item
  Il faudrait une description plus précise des variables :
  \texttt{type\ de\ suivie}, \texttt{type\ de\ protocole},
  \texttt{version\ protocole}
\item
  Pour une même variable, il peut y avoir des noms différents :
  \texttt{Libellé\ Sortie}= \texttt{libellé\ sortie}
\end{itemize}

\hypertarget{le-type-de-donnuxe9es-uxe0-exploiter}{%
\subsubsection{\texorpdfstring{\textbf{2. Le type de données à
exploiter}}{2. Le type de données à exploiter}}\label{le-type-de-donnuxe9es-uxe0-exploiter}}

\begin{itemize}
\tightlist
\item
  \emph{Brouillon} : La fiche est incomplète
\item
  \emph{Terminé} : La fiche est terminée par l'opérateur
\item
  \emph{Validé} : La fiche est validée par un référent local (structure)
  ou national
\end{itemize}

\emph{RG :} Pour la description des données j'utilise les données
validées. \emph{AG :} Il faut vérifier les recommandations d'analyse des
données (fichier envoyé par \emph{LB})

\hypertarget{mise-en-place-du-comituxe9-de-pilotage-copil}{%
\subsubsection{\texorpdfstring{\textbf{3. Mise en place du Comité de
pilotage
COPIL}}{3. Mise en place du Comité de pilotage COPIL}}\label{mise-en-place-du-comituxe9-de-pilotage-copil}}

\hypertarget{participants}{%
\paragraph{\texorpdfstring{\emph{3.1
Participants}}{3.1 Participants}}\label{participants}}

\begin{itemize}
\tightlist
\item
  \textbf{DIRMM} Juliette AMAT -
  \href{mailto:juliette.amat@@developpement-durable.gouv.fr}{\nolinkurl{juliette.amat@@developpement-durable.gouv.fr}}
\item
  \textbf{DREAL} Normandie vérif avec Sandrine ROBBES ou convention
\item
  \textbf{OFB} Sebastien BITON -
  \href{mailto:sebastien.biton@@ofb.gouv.fr}{\nolinkurl{sebastien.biton@@ofb.gouv.fr}}
\item
  \textbf{AESN} Yann JONCOURT
\item
  \textbf{AEAP} Jean PRIGEL -
  \href{mailto:j.prygiel@@eau-artois-picardie.fr}{\nolinkurl{j.prygiel@@eau-artois-picardie.fr}}
\item
  \textbf{AELB} Philippe FERAT
\item
  \textbf{URCPIE} Léa BRIEAU -
  \href{mailto:observatoirepapl@@urcpie-normandie.com}{\nolinkurl{observatoirepapl@@urcpie-normandie.com}}
\item
  \textbf{OFB ESTAMP} Elodie GAMP -
  \href{mailto:elodie.gamp@@ofb.gouv.fr}{\nolinkurl{elodie.gamp@@ofb.gouv.fr}}
\item
  \textbf{OFB ESTAMP} Guilhem AUTRET -
  \href{mailto:guilhem.autret@@afbiodiversite.fr}{\nolinkurl{guilhem.autret@@afbiodiversite.fr}}
\item
  \textbf{Réseau Littorea} Franck DELISLE et Sarah OLIVIER -
  \href{mailto:reseau.littorea@@gmail.com}{\nolinkurl{reseau.littorea@@gmail.com}}
\item
  \textbf{DDTM Manche} Remplaçant Célia D'Hervé
\item
  \textbf{DDTM Calvados} ?
\item
  \textbf{DDTM Côtes d'Armor} Valérie JAOUEN
\end{itemize}

\hypertarget{organisation}{%
\paragraph{\texorpdfstring{\emph{3.2
Organisation}}{3.2 Organisation}}\label{organisation}}

\begin{itemize}
\tightlist
\item
  Présentation du projet OBADE
\item
  Description de la base
\item
  Présentation des outils existants
\item
  Quelles questions à répondre pour OBADE sur \textbf{les données
  d'usage}
\end{itemize}

\hypertarget{divers}{%
\subsubsection{\texorpdfstring{\textbf{4.
Divers}}{4. Divers}}\label{divers}}

\emph{RG :} Pour le COPIL création d'un modèle de document d'analyse
selon le fichier (.xlsx) développé par \emph{LB}. (\emph{e.g.} Côtes
d'Armor tous sites confondus)

Conformité des paniers: Difficile à analyser car données qui n'ont pas
les mêmes unités (poids, volume, nombre d'individus)

Voir l'intégration de la base SIG dans le site du projet OBADE
\href{https://lizmap.afbiodiversite.fr/aamp/afb_mer/index.php/view/map/?repository=bdestampbasesig\&project=ref_met_bdestamp_ref_geo_pol}{lien}

\end{document}
